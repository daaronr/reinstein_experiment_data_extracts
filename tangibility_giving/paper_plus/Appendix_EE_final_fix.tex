%% LyX 1.6.7 created this file.  For more info, see http://www.lyx.org/.
%% Do not edit unless you really know what you are doing.
\documentclass[12pt,english,authoryear,review]{elsarticle}
\usepackage{mathptmx}
\usepackage[T1]{fontenc}
\usepackage[latin9]{inputenc}
\usepackage[letterpaper]{geometry}
\geometry{verbose,bmargin=3cm,lmargin=3cm}
\usepackage{array}
\usepackage{textcomp}
\usepackage{graphicx}
\usepackage{setspace}
\doublespacing

\makeatletter

%%%%%%%%%%%%%%%%%%%%%%%%%%%%%% LyX specific LaTeX commands.
%% Special footnote code from the package 'stblftnt.sty'
%% Author: Robin Fairbairns -- Last revised Dec 13 1996
\let\SF@@footnote\footnote
\def\footnote{\ifx\protect\@typeset@protect
    \expandafter\SF@@footnote
  \else
    \expandafter\SF@gobble@opt
  \fi
}
\expandafter\def\csname SF@gobble@opt \endcsname{\@ifnextchar[%]
  \SF@gobble@twobracket
  \@gobble
}
\edef\SF@gobble@opt{\noexpand\protect
  \expandafter\noexpand\csname SF@gobble@opt \endcsname}
\def\SF@gobble@twobracket[#1]#2{}
%% Because html converters don't know tabularnewline
\providecommand{\tabularnewline}{\\}

%%%%%%%%%%%%%%%%%%%%%%%%%%%%%% User specified LaTeX commands.
\usepackage{booktabs}
\usepackage{marvosym}
\usepackage{rotating}
\usepackage{pdflscape}
\usepackage{multirow}
\usepackage{lscape}

\makeatother

\usepackage{babel}

\begin{document}

\title{Appendix: Decomposing Desert and Tangibility\textbf{ }Effects in
a Charitable Giving Experiment}

\maketitle

\section{Protocol}


\subsection*{Lab Setup}

There must be two parts to the laboratory. One, the \textquotedblleft{}Inside\textquotedblright{}
is where the subjects sit at their desks/computers, and the other,
the \textquotedblleft{}Outside\textquotedblright{} is where we the
experimenters, meet the subjects at the beginning and end. The Inside
must not be viewable from the Outside, and this must be obvious. We
set up subject computers, and the relevant handouts and numbers on
desks Inside, but the server computer and the \textquotedblleft{}N\textquotedblright{}
z-leaf must be set up Outside. We need to have access to a printer
Outside.

Set number of subjects in Background and Global. Sort clients and
pre-fill envelopes with receipts and money and build three stacks. 


\subsection*{Timing }
\begin{enumerate}
\item Participants meet Outside
\item Give short description of what will happen {[}Briefing{]} 
\item Ask for a volunteer. If there are more, select them by drawing balls
from the Urn. Brief the volunteer
\item Participants draw a number from the box and are advised not to let
us see it but to look at the number \textquotedblleft{}Inside\textquotedblright{}
the lab facility, and report to the desk with that number on it and
follow the instructions. {[}Instructions performance{]} 
\item {[}PERFORMANCE TREATMENTS ONLY{]} Start the WORK Stage
\item {[}CASH TREATMENT ONLY{]} Experimenter screen Prompts Subject number
and payment. Look up subject number and computer number in subject
table and put post it notes with subject number on the prepared envelopes
{[}ENVELOPE{]}

\begin{enumerate}
\item Take pre-filled envelopes and put in: If subject shock=1: 3 donation
envelopes If subject shock=0: 2 donation envelopes (Brot and WWF)
and put subject number on post-it on the big brown envelope Prompt
the volunteer to come out. Hand over the box with envelopes and instruct
the volunteer to distribute the envelopes to the tables 10.{[}on screen{]}
Instruct subjects to open the envelopes, count the money and enter
the amount 
\end{enumerate}
\item DONATION PHASE \textendash{} Donation
\item {[}CASH TREATMENT ONLY{]} Screen that reports payments tells subjects
to put the amount they promised to donate into the appropriate plain
white envelopes. 
\item After Questionnaire

\begin{enumerate}
\item {[}CASH TREATMENTS ONLY{]} Subjects are instructed to collect their
belongings and get up from their desks. They put the SPENDEN {[}donation{]}
envelopes into the SPENDEN {[}donations{]} box and the BELEG {[}document
envelopes{]} into the BELEG {[}document{]} Box. Volunteer makes sure
that all are ready the subjects are to come \textquotedblleft{}Outside\textquotedblright{}
to meet us. Volunteer opens the \textquotedblleft{}SPENDEN\textquotedblright{}
{[}Donations{]} box and adds the actual donations. Volunteer observes
that experimenter donated the correct amount of money (online, using
credit card) and signs to this effect. We pay volunteer, and volunteer
signs receipt of this. Volunteer also signs Volunteer Witness Form
saying that \textquotedblleft{}I witnessed that experimenters made
{[}AMOUNTS HERE{]}� payments to charities. This payment equaled the
total of the actual subject contributions.''
\item {[}ENTITLEMENT TREATMENTS ONLY{]} Subjects are instructed to collect
their belongings and get up from their desks. They put the BELEG {[}documents{]}
into the BELEG {[}documents{]} Box. Volunteer makes sure that all
are ready the subjects are to come \textquotedblleft{}Outside\textquotedblright{}
to meet us. Experimenters add up the donations recorded by ztree and
show them to volunteer. Volunteer observes that experimenter donated
the correct amount of money (online, using credit card) and signs
to this effect. We pay volunteer, and volunteer signs receipt of this.
Volunteer also signs Volunteer Witness Form saying that \textquotedblleft{}I
witnessed that experimenters made {[}AMOUNTS HERE{]}� payments to
charities.''
\end{enumerate}
\end{enumerate}

\subsection*{Records and book-keeping}

Receipt forms (and Volunteer Witness Form) and records of donations
(email response from charity, credit card record, letters from charity
when they arrive) will be sent to the graduate school administrator
at [Names of universities hidden]. Subjects are told that they can contact
the administrator {[}name provided to the subjects{]} if they are
still skeptical and want to verify the (total, per session) donations
made. \clearpage


\section{Description of charities\label{sec:Description-of-charities}}


\subsection*{Brot f�r die Welt (Bread for the World)}

Brot f�r die Welt is a development organisation by the church founded
in 1959 in Berlin. It is supported by all the country's Protestant
and independent churches. The management of the organisation \textquotedbl{}Bread
for the World\textquotedbl{} is located at the Diakonisches Werk der
EKD eV, which is the legal entity of action. The annual fund-raising
starts on the first Advent, the beginning of the liturgical year.
Every action is under a particular theme, which will indicate specifically
funded projects. Most development projects are assigned to various
program topics. In 2007 \textquotedbl{}Brot f�r die Welt\textquotedbl{}
mainly promoted measures to ensure food security and access to basic
services such as education and health. Other supported areas are peacekeeping
and democracy promotion, and the fight against HIV / AIDS. As of 2005,
they have received over 1.6 billion Euro in donations for aid projects
in Africa, Asia, Latin America and for several years in Eastern Europe.
In 2006 Bread for the World received donations amounting to 51.5 million
euros.


\subsection*{WWF}

The WWF, the World Wide Fund For Nature, is one of the largest international
nature conservancy organisation worldwide. It was founded in Switzerland
in 1961 as World Wildlife Fund. The WWF wants to halt the worldwide
destruction of nature and create a future where humanity lives in
harmony with nature. The WWF stands up for: conserving ecological
diversity, the sustainable use of natural resources, and the reduction
of pollution and harmful consumer behavior. Over the years the areas
of expertise have grown from pure species preservation: Now general
topics of protecting the environemnt and climate change are on the
agenda of the WWF.


\subsection*{Deutsches Rotes Kreuz (German Red Cross)}

The German Red Cross is committed to life, health, welfare, protection,
peaceful coexistence and the dignity of all people. All people in
need have the same entitlement to assistance, without regard to nationality,
race, religion, sex, social status or political conviction. The DRC
offers help solely on the degree of need and the urgency of the assistance.
The voluntary assistance is used to restore the powers of self-help
for people in need. The DRC will offer all services that are necessary
to fulfill our mandate. They should meet the highest standards and
quality requirements. In fulfillment of our own objectives , the DRC
cooperates with all institutions and organizations in state and society
that can be helpful and / or have similar objectives. However, we
are preserving our independence. We respond to competition from others
by improving the quality of our assistance, but also its economic
viability.


\section{Screenshots of experimental stages}

\begin{figure}[H]
\caption{The real effort task}

\centering{}\includegraphics[scale=0.3]{Task}
\end{figure}

%
\begin{figure}[H]
\caption{Promised Payments}


\begin{centering}
\includegraphics[scale=0.3]{PromisePay}
\par\end{centering}

\clearpage 

\medskip{}


\centering{}You obtain 10.00� for this experiment. Please press OK.
\end{figure}


%
\begin{figure}[H]
\caption{Cash Payments}


\begin{centering}
\includegraphics[scale=0.3]{CashPay}
\par\end{centering}

\medskip{}


You obtain 10.00� for this experiment. 

The volunteer will now go to the outer part of the lab to get the
envelopes and distribute them. Please remain seated in the meanwhile
and do not talk to your neighbors. 

As soon as you receive the money, please count it. 

Press OK after you have counted the money and signed the receipt.
\end{figure}


%
\begin{figure}[H]
\caption{Donation Stage}


\begin{centering}
\includegraphics[scale=0.3]{Donation_Stage}
\par\end{centering}

\medskip{}

\begin{quotation}
Your earnings: �10.00

{[}Donation Decisions{]}

Your donation will be transferred under the supervision of the volunteer
to the respective organizations.
\end{quotation}

\end{figure}

\clearpage

\newpage

\section{Extensive margin regressions\label{sec:Extensive-margin-regressions}}

\begin{table}

\caption{Extensive margin: Probit regressions of Positive Donation\label{probits}}

\include{ProbitTreat2Probitsdr_ok}

Note: Linear probability models (available by request) yield nearly identical results.
\end{table}

\clearpage
\newpage 

\section{Robustness checks\label{sec:Robustness-checks}}

Table \ref{tab:Ratio-donated} presents the results of the regression
on the proportion of income donated, using Papke-Wooldridge estimator
for fractional response variables. The coefficient on cash payments
is still significant and negative, while the coefficient on performance
pay loses significance when adding additional controls, but the coefficient
itself does not change.

\begin{table}
\caption{Ratio of income donated (Papke-Wooldridge estimator)\label{tab:Ratio-donated}}

\include{ShareTreat}
\end{table}


\clearpage

In table \ref{tab:Regressions-by-charity} we run the regressions
from main text table 3 column 5 split by charity. We find a similar
pattern for the cash treatments over all charities, although the coefficients
on cash are are only significantly negative for WWF and DRK. 

%
\begin{table}[H]
\caption{Poisson regressions by charity\label{tab:Regressions-by-charity}}

\include{PoissonRobustCheck}
{\footnotesize Marginal effects reported. Constant dropped.}
\end{table}

\clearpage


\textbf{Session and time of day effects }

As table \ref{tab:sessionstrtmts} illustrates, our treatments are also not perfectly
balanced over time:

%
\begin{table}[H]
\caption{Schedule of sessions and treatments}\label{tab:sessionstrtmts}
\begin{tabular}{|c|c|c|c|c|c|}
\hline 
Date (m/d/yr) & Time & \multicolumn{4}{c|}{Subjects in treatments}\tabularnewline
\hline
\hline 
 &  & {\footnotesize Account/Random} & {\footnotesize Cash/Random} & {\footnotesize Account/Performance} & {\footnotesize Cash/Performance}\tabularnewline
\hline 
{\footnotesize 10/27/08} & {\footnotesize 9:50} & {\footnotesize 10} &  &  & \tabularnewline
\hline 
 & {\footnotesize 11:49} & {\footnotesize 10} &  &  & \tabularnewline
\hline 
{\footnotesize 10/28/08} & {\footnotesize 9:51} &  &  & {\footnotesize 10} & \tabularnewline
\hline 
 & {\footnotesize 11:43} &  &  &  & {\footnotesize 9}\tabularnewline
\hline 
{\footnotesize 02/25/09} & {\footnotesize 9:30} & {\footnotesize 18} &  &  & \tabularnewline
\hline 
 & {\footnotesize 11:56} &  &  & {\footnotesize 15} & \tabularnewline
\hline 
{\footnotesize 02/26/09} & {\footnotesize 11:25} &  & {\footnotesize 18} &  & \tabularnewline
\hline 
{\footnotesize 03/02/09} & {\footnotesize 10:42} &  &  &  & {\footnotesize 18}\tabularnewline
\hline 
 & {\footnotesize 12:08} &  & {\footnotesize 10} &  & \tabularnewline
\hline 
{\footnotesize 10/30/09} & {\footnotesize 10:11} &  & {\footnotesize 18} &  & \tabularnewline
\hline 
 & {\footnotesize 11:33} &  &  &  & {\footnotesize 18}\tabularnewline
\hline 
 & {\footnotesize 13:07} & {\footnotesize 18} &  &  & \tabularnewline
\hline 
 & {\footnotesize 14:38} &  &  & {\footnotesize 18} & \tabularnewline
\hline
\end{tabular}
\end{table}
\clearpage

To test for session-specific effects, we report regressions with standard
errors clustered by session, and controls for time-of-day and time-of-year
effects; our results are robust to all of these. We divide our session
times into three categories: 9.30-10:30 am, 10:31-12 noon, and afternoon
(12:01-14:38pm). The regressions below control for all of these {}``time
dummies'', and they are not jointly significant. We also divide our
sessions into four {}``sets'': those run in October of 2008, those
run in February and March of 2009, and those run in October 2009.
Again, these dummies are not jointly significant in any of the regressions
below. 

%
\begin{table}[h]
\caption{Poisson and OLS regression on total donations\label{tab:Poisson-and-OLS-1}}
\begin{tabular}{lcccc}
 & \multicolumn{2}{c}{ } & \multicolumn{2}{c}{}\tabularnewline
\hline
\hline 
 &  &  & \multicolumn{2}{c}{Gender contr.}\tabularnewline
\hline
 & \multicolumn{1}{c}{(1)} & \multicolumn{1}{c}{(2)} & \multicolumn{1}{c}{(3)} & \multicolumn{1}{c}{(4)}\tabularnewline
 & \multicolumn{1}{c}{Psn.} & \multicolumn{1}{c}{OLS} & \multicolumn{1}{c}{Psn.} & \multicolumn{1}{c}{OLS}\tabularnewline
\hline
Pay cash  & -0.78{*}{*}  & -0.84{*} & -0.82{*}{*} & -0.89{*}{*}\tabularnewline
 & (0.20)  & (0.34) & (0.21)  & (0.34) \tabularnewline
Pay by performance  & -0.53{\footnotesize +}  & -0.54 & -0.57{*}  & -0.58{\footnotesize +}\tabularnewline
 & (0.32)  & (0.35) & (0.28)  & (0.35) \tabularnewline
Cash $\times$ performance & 0.42  & 0.44 & 0.55  & 0.56 \tabularnewline
 & (0.58)  & (0.50) & (0.59)  & (0.50) \tabularnewline
Third charity  &  &  & 0.41 & 0.26\tabularnewline
 &  &  & (0.4)  & (0.25)\tabularnewline
Stake: 7.5  &  &  & -0.09  & -0.084\tabularnewline
 &  &  & (0.52)  & (0.30)\tabularnewline
Stake: 10  &  &  & 0.16  & 0.100\tabularnewline
 &  &  & (0.45)  & (0.30) \tabularnewline
Female  &  &  & 0.66{*} & 0.53{*} \tabularnewline
 &  &  & (0.29)  & (0.25) \tabularnewline
Time dummies <Chi-sq>/{[}F-test{]} & <0.23> & {[}0.12{]} & <0.54> & {[}0.26{]}\tabularnewline
\{P-value of test\} & \{0.89\} & \{0.89\} & \{0.76\} & \{0.77\}\tabularnewline
{}``Set'' dummies <Chi-sq>/{[}F-test{]} & <0.85> & {[}0.48{]} & <1.41> & {[}0.59{]}\tabularnewline
\{P-value of test\} & \{0.65\} & \{0.63\} & \{0.50\} & \{0.57\}\tabularnewline
\hline 
 Observations  & 190  & 190  & 190  & 190 \tabularnewline
$R^{2}$  &  & 0.048  &  & 0.079 \tabularnewline
Pseudo $R^{2}$ & 0.034  &  & 0.057  & \tabularnewline
\hline
\multicolumn{5}{l}{{\footnotesize Standard errors in parentheses, reported clustered
by session for OLS.}}\tabularnewline
\multicolumn{5}{l}{{\footnotesize + p<0.10, {*} p<0.05, {*}{*} p<0.01 for tests using
standard errors clustered by session }\emph{\footnotesize for all
columns.}}\tabularnewline
\multicolumn{5}{l}{{\footnotesize All regressors are dichotomous (0,1) variables, dy/dx
for discrete change of dummy variable reported, }}\tabularnewline
\multicolumn{5}{l}{{\footnotesize Marginal effects evaluated at Account/Random, Female,
Stake = 7.5, two charity choice set.}}\tabularnewline
\end{tabular}
\end{table}

\clearpage 

\bigskip 


\textbf{Selection on performance differences }

It is conceivable that those who do better on the task earn more,
and these people might be less generous on average. This {}``selection''
might cause us to falsely attribute this to a desert effect -- when
we compare the high earners to those with high randomly-assigned endowments,
the former would tend to give less. As evidence against this, we find
the same effect across all stake sizes (results available by request).
As payments when players get a tie score are randomly assigned, we
can also control for the absolute level of performance (regression
tables available by request). Adding a control for the ``number
of correctly solved sums'' to the regressions in main text table 3
barely alters any of coefficients, and the coefficient on this control
variable is tiny, insignificant, and tightly bounded around zero (e.g.,
if we add this variable to the first column of main text table 3
its coefficient has a 95\% confidence interval of -0.08, 0.05) (table available by request). 

This result supports \citet{cherry2005impact}, who write ``the selection of high and low endowment dictators in the earned and windfall treatments differ (exam score versus random), which may raise questions
of sample selection, but previous research using this selection method has found this is not a significant concern.'' \clearpage


\section{Statistics on subjects' charity preferences and attitudes toward charities \label{sec:charpref}}

\begin{table}[h]
\caption{Percentage of subjects who donated to specific charity.}
\begin{tabular}{lccc||c} 
\hline \hline Choice Set  & BfdW & DRK & \multicolumn{2}{c}{WWF}\tabularnewline
\hline Two Charities & 33.39\% & - & \multicolumn{2}{c}{30.69\% }\tabularnewline 
Three Charities & 47.32\%  & 41.67\% & \multicolumn{2}{c}{39.71\%}\tabularnewline
Overall & 39.68\% & 41.67\% & \multicolumn{2}{c}{34.65\%}
\tabularnewline \hline \end{tabular}
\end{table}

\begin{table}[H]
\caption{Answer to the question: {}``I trust that the charity uses the money
as they state'' (percentages giving each response)}

\centering{}\begin{tabular}{lccc}
\hline 
 &  WWF & BfdW & DRK\tabularnewline
\hline
\hline 
Don't agree at all & 0.00 & 0.00 & 2.11\tabularnewline
Don't agree & 10.53 & 8.42 & 8.95\tabularnewline
No opinion & 11.58 & 14.21 & 12.63\tabularnewline
Agree & 59.47 & 56.84 & 52.11\tabularnewline
Strongly agree & 14.21 & 15.79 & 14.74\tabularnewline
No comment & 4.21 & 4.74 & 9.47\tabularnewline
\hline
\end{tabular}
\end{table}

\begin{table}[H]
\caption{Answer to the question: {}``What percentage of the money donated
reaches the needy'' (percentage giving each response)}

\centering{}\begin{tabular}{lccc}
\hline 
 &  WWF & BfdW & DRK\tabularnewline
\hline
\hline 
less than 50\%  & 26.32 & 24.74 & 26.84\tabularnewline
50-75\% & 40.53 & 40.53 & 34.21\tabularnewline
75-90\% & 27.37 & 30.53 & 31.05\tabularnewline
90-100\% & 5.79 & 4.21 & 7.89\tabularnewline
\hline
\end{tabular}
\end{table}

\clearpage

\section{Endowment sizes by treatment \label{sec:endowmentsizes}}

\begin{table}
\caption{Endowments by treatment (realized)\label{AllocationSizeVSdimension_dred}}
\include{AllocationSizeVSdimension}
\end{table}

Table \ref{AllocationSizeVSdimension_dred} summarizes the endowments by treatment, giving the
frequencies of each endowment, and the mean and standard deviation
of the endowments. The imbalance resulted from {}``no-shows'' in
particular sessions. However, the regression analysis controls for
(and finds no evidence of) a {}``stake size'' effect on donations,
and the nonparametric analysis involves a bootstrap that, in each
repetition, balances the the endowments by treatment. 
\clearpage

\section*{References}
\bibliographystyle{chicago}
\bibliography{housemoney_lit}

\clearpage
\end{document}
